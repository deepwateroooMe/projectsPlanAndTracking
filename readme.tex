% Created 2023-12-21 Thu 16:28
\documentclass[9pt, b5paper]{article}
\usepackage{xeCJK}
\usepackage[T1]{fontenc}
\usepackage{bera}
\usepackage[scaled]{beraserif}
\usepackage[scaled]{berasans}
\usepackage[scaled]{beramono}
\usepackage[cache=false]{minted}
\usepackage{xltxtra}
\usepackage{graphicx}
\usepackage{xcolor}
\usepackage{multirow}
\usepackage{multicol}
\usepackage{float}
\usepackage{textcomp}
\usepackage{algorithm}
\usepackage{algorithmic}
\usepackage{latexsym}
\usepackage{natbib}
\usepackage{geometry}
\geometry{left=1.2cm,right=1.2cm,top=1.5cm,bottom=1.2cm}
\usepackage[xetex,colorlinks=true,CJKbookmarks=true,linkcolor=blue,urlcolor=blue,menucolor=blue]{hyperref}
\newminted{common-lisp}{fontsize=\footnotesize} 
\author{deepwaterooo}
\date{\today}
\title{Winter 3 Weeks Holidays: Basic Plan + Tracking}
\hypersetup{
  pdfkeywords={},
  pdfsubject={},
  pdfcreator={Emacs 29.1 (Org mode 8.2.7c)}}
\begin{document}

\maketitle
\tableofcontents


\section{Basic Plan: \textbf{【亲爱的表哥的活宝妹,任何时候,亲爱的表哥的活宝妹就是一定要、一定会嫁给活宝妹的亲爱的表哥!!!爱表哥,爱生活!!!】}}
\label{sec-1}
\begin{itemize}
\item \textbf{【亲爱的表哥的活宝妹,任何时候,亲爱的表哥的活宝妹就是一定要、一定会嫁给活宝妹的亲爱的表哥!!!爱表哥,爱生活!!!】}
\end{itemize}
\subsection{\textbf{【夏天两三个项目:】}}
\label{sec-1-1}
\begin{itemize}
\item 项目暂停、时间再过四个月,亲爱的表哥的活宝妹,明天再看,能否无师自通、自己开各种窍窍呢?!!明天再看暂停了四个月的夏天两三个项目,亲爱的表哥的活宝妹,会否如先前算法般、再次体会、项目上的成长进步?
\begin{itemize}
\item 时间转得狠快,转眼四个月眨眼就过去了,亲爱的表哥的活宝妹,夏天的两三个项目,就因为亲爱的表哥的活宝妹先前完全不懂的 linux 操作系统,被放去休息四个月。
\item 不是亲爱的表哥的活宝妹,挫败,只是先前完全不会操作系统的亲爱的表哥的活宝妹,没能挤出时间精力去弄。
\item 亲爱的表哥的活宝妹,自两年前的 8 月,开始真正从头、从第一个题目写算法,那个时候,实在是傻傻不知道怎么想题目;可是半年、大半年后,亲爱的表哥的活宝妹,只要休息好,只要能够自己把思路想明白,亲爱的表哥的活宝妹,就是有绝对的自信,一定可以修改掉过程中边边角角的小错误与偏差,把题目完全解出来、通过所有测试。 \textbf{亲爱的表哥的活宝妹,自己,从算法的练习里,建立过编程与解题的极彻底、让亲爱的表哥的活宝妹自己动容的自信。亲爱的表哥的活宝妹,觉得,时间是太神奇的催化剂,亲爱的表哥的活宝妹也女大十八变,亲爱的表哥的活宝妹的算法,原来也可以、也会无师自通!!}
\item 时间转得狠快,转眼四个月眨眼就过去了,亲爱的表哥的活宝妹,夏天的两三个项目,暂停四个月,亲爱的表哥的活宝妹,现在要继续弄它们前的第一想法,这三个周一定会再回去看,写得完善些。 \textbf{动手前、下口前,这一次,亲爱的表哥的活宝妹,先想,是否会体会如先前、去年工作后体会过的,亲爱的表哥的活宝妹,又女大十八变般,又无师自通了呢?接下来几天检测一下!!}
\end{itemize}
\end{itemize}
\subsection{\textbf{【Mac-M1-M2 项目环境配置:】} 主要是 \textbf{C\# 游戏框架相关的项目、配置需求}}
\label{sec-1-2}
\begin{itemize}
\item 亲爱的表哥的活宝妹的M2, 上半年还没能真正适应,各种项目的运行,基本仍然是在破烂快死掉的 Windows 10 系统上。
\item 主要原因也是: \textbf{中文社区用苹果的人少,中文社区的必要科普、普及不够。英文社区,在亲爱的表哥的活宝妹这里,可能,还存在一定程度上的语言障碍、与消化理解困难。}
\item 那个时候,亲爱的表哥的活宝妹, \textbf{不懂得、不知道、该怎么配置必要的项目环境。}
\item 现在,买电脑快1 年了(10 个月),亲爱的表哥的活宝妹,觉得, \textbf{现在,接下来三个周,应该是时候,广泛搜索,为亲爱的表哥的活宝妹自己科普、学习与深入,Mac-M1-M2 里,亲爱的表哥的活宝妹所需要的、必要的项目运行环境配置。}
\item 若能够 \textbf{把 Rime M1M2 环境下的输入法,自己编译安装进电脑,也算是,一个大进步} 。先前,亲爱的表哥的活宝妹,连 emacs-pyim-librime 动态库都建不出来。。。
\end{itemize}
\subsection{\textbf{【实现几个小项目:】} 推 deepwateroooMe-github 上面去}
\label{sec-1-3}
\begin{itemize}
\item 去年离职前,亲爱的表哥的活宝妹的 deepwaterooo-github 帐户,被他们锁死弄没了。亲爱的表哥的活宝妹,不得不重建一个 deepwateroooMe-github 帐户,可是仓库少了太多。 \textbf{亲爱的表哥的活宝妹先前的 deepwaterooo-github 帐户,纪录着亲爱的表哥的活宝妹专业成长足迹的、几乎所有仓库,都不见了。极其可惜。}
\item 亲爱的表哥的活宝妹,上个周,因为代课老师要求,打算 fork-xv6-public 时,发现无法顺利登录 deepwateroooMe-github 帐户。担心过,不过还好,今天看它还是活的!
\item 亲爱的表哥的活宝妹,觉得 \textbf{上面的项目太少了, 50 个仓库,还有狠多是 fork 他人的} 。觉得 \textbf{只看相对复杂一点儿的大型游戏框架,就显得自己设计与实现的项目太少,而真正的基本功训练也不可少。想要再添加、调序与实现几个项目,推上去。} 现在只是不清楚,是实现(哪怕是修改他人的原始 naive 版本项目)C\# 游戏项目,还是设计实现几个安卓项目?想想,希望这三个周添加几个。
\end{itemize}
\subsubsection{\textbf{C\# 游戏项目} :可能找 github 上的幼稚项目,改成亲爱的表哥的活宝妹,希望的样子!!}
\label{sec-1-3-1}
\subsubsection{\textbf{安卓项目} :就需要亲爱的表哥的活宝妹,自己设计、实现,难度上应该到中高层次,才可以。}
\label{sec-1-3-2}
\begin{itemize}
\item 当学习、补习了先前欠缺的 \textbf{【Linux 操作系统】} 相关的, \textbf{大型安卓项目,涉多进程、多进程跨进程通信IPC 等、多分布、相对独立的功能模块的设计、实现、与【冷温热启动、加载延迟等】性能优化} ,就成为亲爱的表哥的活宝妹现在,可以理解、消化,和进阶的练手重点。先前只是被暴露、和接触到过这些术语,现在应该是真正深入弄明白的时候,甚至实现练手一下。
\begin{itemize}
\item 先前接触过 \textbf{【多进程应用】可能会涉及的、因多进程而引入的各种问题} (甚至静态成员变量的值、单例模式不再单例?,因为是多进程跨进程应用,而变得不固定?!) 。但是如果能够真正设计,实现、或是改装一个它人的极初版本,体会收获会狠多!
\end{itemize}
\item 同样的,对于安卓百尺杆头、更进一步,也需要更 \textbf{广泛使用 Service, ContentProvider 等,和深入理解 Activity, Context 等的底层原理} 。经典的 ActivityManagerService?ActivityServiceManager? 原理?
\item 当工作过5 个月左右,当接触车载安卓系统,当深入底层学习过Linux ARM 系统的设计与实现几大主要功能模块, \textbf{安卓系统的 AppLauncher 诸应用的加载管理应用?, SystemUI 系统视图,等相对底层,该成为亲爱的表哥的活宝妹,现在可以着手弄懂,甚至实现练手一下的QEMU 模拟的【Linux 系统】} :ARM Linux 系统,亲爱的表哥的活宝妹,见过了 ! 
\begin{itemize}
\item 脑袋里面的设计与实现过程是:回想车载系统里,Linux 系统的应用层主要设计与实现,对比亲爱的表哥的活宝妹的舅舅刚教授过的ARM Linux 系统,来借助QEMU 组装一个麻雀虽小五脏俱全的?
\end{itemize}
\item 安卓再然后:SDK, NDK, PushService? 安卓涉及网络的部分、推送相关的,现在比较流行用哪些?亲爱的表哥的活宝妹,比较喜欢如 Emacs-org-mode 里般,大家在用什么,亲爱的表哥的活宝妹比较喜欢新新人类,跟上大家的使用步伐!
\item 当接触了Unity 游戏引擎后,相关联的: 
\begin{itemize}
\item \textbf{【构建游戏流程相产的,一整套】} :这些,春上自己折腾过,知道这些牵连,知道问题存在,知道解题方向,细节要一一总结。
\begin{itemize}
\item Unity 游戏引擎里,直接构建,哪些好与不好?
\item 当需要导出到安卓打包,安卓SDK 的接入,几种方法,关键知识点
\end{itemize}
\item \textbf{【Unity 游戏引擎,与安卓相交互的,必要知识点:】}
\begin{itemize}
\item 与安卓SDK 相关联的,Broadcast 等实现上的几种不同方法【回字的四样写法。。。】
\end{itemize}
\end{itemize}
\item 【安卓Generational Heap Memory 内存管理模型】:深入学习和总结一下
\end{itemize}
\subsection{\textbf{【google-play 上,希望尽快上1个小游戏,走一遍流程:】}}
\label{sec-1-4}
\subsection{\textbf{【Unity 里的学习,确定一个方向】} :【接下来的感兴趣的领域,确定一两个、几个方向】}
\label{sec-1-5}
\begin{itemize}
\item Unity 游戏引擎,涉及到的知识面比较广了。上半年主要弄了些游戏框架里网络相关的部分,还不一定消化理解得透彻。亲爱的表哥的活宝妹,应该 \textbf{需要,为自己在个这个引擎上的学习,确定几个自己能够理解消化得透彻、真正感兴趣、甚至将来可以作为自己工作内容的方向。}
\begin{itemize}
\item \textbf{【ARM-Linux 系统,与内存管理】} :这个学期,亲爱的表哥的活宝妹的【操作系统】相关的基础知道补习,还算是狠彻底的,主要是借助亲爱的表哥的活宝妹的舅舅的课上的作业与考试。
\item \textbf{【安卓内存优化】,与【Unity 内存管理,与优化】} 相关的,可以再复习一遍,复习,总结,也寻找自己感兴趣的版块与方向。
\end{itemize}
\end{itemize}
\subsection{\textbf{【被广泛接纳、使用的、第三方库,的学习:】} 需要涉及、遍布一定程度上的广度,扩充知识面。}
\label{sec-1-6}
\begin{itemize}
\item \textbf{【emacs-org-mode 工具的启发】} :
\begin{itemize}
\item 上半年的 pyim, 是多少年来,亲爱的表哥的活宝妹,第一次真正实现 emacs 里、自己的中文输入。多少年前,IDE 还不先进不完善,亲爱的表哥的活宝妹自己安装 linux 系统,自己配置 linux系统里 latex 相关的环境,字体等,那个时候,它是每台新电脑、笔记本最头痛的难题,不死去活来折腾好几天,亲爱的表哥的活宝妹,都配置不出来。现在都“傻瓜相机”化了,不再头痛。
\item \textbf{亲爱的表哥的活宝妹,是如何,把自己变成 emacs 文本编辑器 org-mode 相关配置的,【中高端】水平的?}
\item 不是想骄傲,为自己迷惑、寻找方向时,提供一个视角。 \textbf{在安卓,C\# 游戏相关版块,亲爱的表哥的活宝妹,要如何借鉴,这些几往的、过往的经验与教训,能够在可以找到的有兴趣的方向上,学习得更深入一点儿?} 最近可以多想想这个问题。
\end{itemize}
\item \textbf{【安卓】} :
\begin{itemize}
\item \textbf{【图库 glide?】} :去年秋天?,他们提过一次,安卓图片库 glide? 里,关于动漫相关的功能模块的完善,可是,那个时候,亲爱的表哥的活宝妹,草草了结了小测试项目,没能理解透彻,或是受到启发,没兴趣。回来可以再看一次。
\item \textbf{【相机、视频相关的】} ,中文社区,感觉流行过了,不再热了。可是基本功,可以自己学习一下
\end{itemize}
\item C\# 游戏相关:感觉,就是那些,各种不同的, \textbf{游戏框架} 了?
\end{itemize}
\subsection{\textbf{【游戏:苹果端,打包、构建流程,熟悉掌握】} :}
\label{sec-1-7}
\begin{itemize}
\item 亲爱的表哥的活宝妹,先前,只做安卓端。现在,基本环境都具备,是否应该也熟悉一下游戏苹果端的构建、打包相关的流程。毕竟,弄个游戏出来只走安卓端,苹果端空着,感觉狠浪费,捡起来也不太难。。
\end{itemize}
\subsection{\textbf{【Emacs org-mode IEEE article automate configurations:】}}
\label{sec-1-8}
\begin{itemize}
\item \textbf{中文社区,科研技术落后,科普就更少或不存在。} emacs 这类命令式编辑器也不流行,用 emacs-org-mode 来配置 IEEE-article 自动化的,就几乎找不到相关的配置。
\item 英文社区有、存在,但是真正牛的分享贴,不包括所有细节;亲爱的表哥的活宝妹,有兴趣想配置,可是某些边边角角找不到参考,倾向于,可能会【行百里者,半于九十】。
\item \textbf{这三个周里,得真正花些时间、精力,才能把这个真正配置出来,到可以完全使用 emacs org-mode 来 100\% 完成所有 summary, 而不需要使用 latex 的程序。}
\item 亲爱的表哥的活宝妹, \textbf{“放养长大”的,自学成长。亲爱的表哥的活宝妹,同样恃才傲物,亲爱的表哥的活宝妹就算体育 30 分只打只得 11 分,全县体育成绩倒数第一,亲爱的表哥的活宝妹,学习成绩、考试照样考全镇第一名!就算亲爱的表哥的活宝妹,还没能配置出 Emacs-org-mode 下的自动化,亲爱的表哥的活宝妹,压根儿就瞧不起 latex 里一个一个字符的敲IEEE-article.}
\item 亲爱的表哥的活宝妹, \textbf{恃才傲物的亲爱的表哥的活宝妹,要么磨刀不误砍柴工,把自动化弄出来;要么,亲爱的表哥的活宝妹就不弄。} 亲爱的表哥的活宝妹,就是这样的 GEEK. 玉不雕刻不成器,亲爱的表哥的活宝妹,恃才傲物的亲爱的表哥的活宝妹,需要活宝妹的亲爱的表哥来亲自教 \textbf{【等亲爱的表哥的活宝妹,如愿嫁给活宝妹的亲爱的表哥了,活宝妹的亲爱的表哥,就可以亲自教亲爱的表哥的活宝妹了!】} !! \textbf{【亲爱的表哥的活宝妹,任何时候,亲爱的表哥的活宝妹就是一定要、一定会嫁给活宝妹的亲爱的表哥!!!爱表哥,爱生活!!!】}
\end{itemize}
\subsection{\textbf{【Linux ARM 最后一个实现项目】}}
\label{sec-1-9}
\begin{itemize}
\item 再多修改一下。亲爱的表哥的活宝妹,昨天,今天,就是体会,真正学习了操作系统后的亲爱的表哥的活宝妹,再看安卓内存模型与管理, so easy\textasciitilde{}!!
\item 所以,如果有时间有兴趣的时候,要把最后一个实现项目,再弄得完整一点儿!! \textbf{【亲爱的表哥的活宝妹,任何时候,亲爱的表哥的活宝妹就是一定要、一定会嫁给活宝妹的亲爱的表哥!!!爱表哥,爱生活!!!】}
\end{itemize}
\subsection{\textbf{【emacs-pyim-librime】} :【TODO】:}
\label{sec-1-10}
\begin{itemize}
\item 亲爱的表哥的活宝妹,现在已经可以构建M1M2 下在的动态包裹。还 \textbf{需要完善一下,中文输入的词库自动化同步} 。哪天花点儿时间,可以简单解决。
\item .emacs.d/pyim/wubi\_ 里,如果亲爱的表哥的活宝妹直接更改词库【可以添加更新后的词库】,那么 pyim 会同时出现更新前被改丢了的,和更新后的词库【这是因为 emacs-pyim 里的 librime 动态库 librime.1.dylib 没能自动更新,存的是更新前的词库】。
\item 自动同步想要,亲爱的表哥的活宝妹,Macbook OS Rime 里,与 emacs-pyim 里,两处仅修改一处,两个地方同步到位,不能要亲爱的表哥的活宝妹,改两次!
\item 所以上面 \textbf{emacs-pyim 里的动态库 librime.1.dylib, 如何才能实现它的自动、实时、动态更新呢?}
\item 如果 emacs-pyim 不使用自己的两个文件、自己的词库,是否可以 Macbook-OS-Rime 与 pyim 共用一个词库?问题涉及 Rime 词库 emacs-pyim 所依赖的动态库 librime.1.dylib 的自动更新!
\item 如果 emacs-pyim 不使用 librime.1.dylib,emacs-pyim 只使用两个文件的词库,指向 Macbook-OS-Rime 配置的地方, emacs-pyim 能否正常运行?问题涉及Macbook-OS-Rime 词库,与Rime 五笔词库,可能不兼容,拼音输入法好像是兼容的,为什么五笔不可以?改天需要再 double-check-confirm 一下这个。如果真是这样,就是 Macbook-OS-Rime 与 emacs-pyim 各占山头、各自为政,不合作!!
\item 试了一下,Macbook-OS-Rime 里同步到一个地方,狠简单,可以同步成功;但是 \textbf{emacs 里,可以成功编译出 liberime-core.so. 可惜,可怜亲爱的表哥的活宝妹,不知道把它摆哪里,怎么用!改天再接着弄。反正目的就是:两处亲爱的表哥的活宝妹只改一处,要两处都能实时更新。【亲爱的表哥的活宝妹,任何时候,亲爱的表哥的活宝妹就是一定要、一定会嫁给活宝妹的亲爱的表哥!!!爱表哥,爱生活!!!】}
\end{itemize}
\subsection{\textbf{【emacs skim legend bug】}: 【状态:【BUG:】不再存在了!!】}
\label{sec-1-11}
\begin{itemize}
\item 亲爱的表哥的活宝妹的放了 1.5 年的 \textbf{【emac skim export BUG:】} ,不曾真正花时间去解决: \textbf{为什么 emacs export pdf SKIM 里永远需要,亲爱的表哥的活宝妹连点三次 enter?得自动化把这三次点击去掉。}
\item 现在,Skim 更新了,终于不用,亲爱的表哥的活宝妹,连点三次 enter 键了?!!
\end{itemize}
\subsection{\textbf{【emacs-csharp-mode】}: 【状态:完成】}
\label{sec-1-12}
\begin{itemize}
\item 这个原理,亲爱的表哥的活宝妹,像是没有弄懂,每次 emacs 的搬迁或是肿么样,这个 csharp-mode 总是出错。上午把这个解决,才能够方便笔记本上运行、阅读各种游戏框架源码。
\item 有点儿没弄清楚,是否 csharp-mode 已经被现在 emacs 版本自带了。亲爱的表哥的活宝妹,不添加自己的配置就一切正常;添加这个模式的自己的配置,反而出错。暂时不管它了,现在没有任何特殊使用需求。等哪天某些功能配置想用、不得不再配置的时候,再弄,效率更高。亲爱的表哥的活宝妹,一大早上的美好时光,就折腾这个破烂 emacs 了。。。
\item 全好了:只要是 emacs 里,自己 melpa 自己安装的,就能够自动适配 arm64, 就不会有 bug. 应该至少当前安装【emacs 被亲爱的表哥的活宝妹笨宝妹再次搬迁前】这个 csharp-mode不出再出错。
\item \textbf{【亲爱的表哥的活宝妹,任何时候,亲爱的表哥的活宝妹就是一定要、一定会嫁给活宝妹的亲爱的表哥!!!爱表哥,爱生活!!!】}
\item \textbf{【亲爱的表哥的活宝妹,任何时候,亲爱的表哥的活宝妹就是一定要、一定会嫁给活宝妹的亲爱的表哥!!!爱表哥,爱生活!!!】}
\item \textbf{【亲爱的表哥的活宝妹,任何时候,亲爱的表哥的活宝妹就是一定要、一定会嫁给活宝妹的亲爱的表哥!!!爱表哥,爱生活!!!】}
\item \textbf{【亲爱的表哥的活宝妹,任何时候,亲爱的表哥的活宝妹就是一定要、一定会嫁给活宝妹的亲爱的表哥!!!爱表哥,爱生活!!!】}
\item \textbf{【亲爱的表哥的活宝妹,任何时候,亲爱的表哥的活宝妹就是一定要、一定会嫁给活宝妹的亲爱的表哥!!!爱表哥,爱生活!!!】}
\item \textbf{【亲爱的表哥的活宝妹,任何时候,亲爱的表哥的活宝妹就是一定要、一定会嫁给活宝妹的亲爱的表哥!!!爱表哥,爱生活!!!】}
\end{itemize}

\section{Weeks Updates + Recordings/Trackings}
\label{sec-2}
\begin{itemize}
\item \textbf{【亲爱的表哥的活宝妹,任何时候,亲爱的表哥的活宝妹就是一定要、一定会嫁给活宝妹的亲爱的表哥!!!爱表哥,爱生活!!!】}
\item 【提交:】每天提交1-2 个仓库,每天提交1-2 次,保障必要的自我监督机制。【亲爱的表哥的活宝妹,任何时候,亲爱的表哥的活宝妹就是一定要、一定会嫁给活宝妹的亲爱的表哥!!!爱表哥,爱生活!!!】
\end{itemize}
\subsection{【12/18: Monday】:}
\label{sec-2-1}
\begin{itemize}
\item 上午,看些安卓内存与 Unity 内存相关;若要睡着,就去看游戏框架源码
\item 折腾 1 小时 emacs-csharp-mode 终于不再打瞌睡【昨天晚上又被破烂房东的噪音机吵一整夜,亲爱的表哥的活宝妹,对它们实在是痛恨得无言语。。。】,再去看看ET 框架,停放了四个月!!
\item 亲爱的表哥的活宝妹,觉得:
\begin{itemize}
\item 以前亲爱的表哥的活宝妹的 csharp-mode 坏掉,亲爱的表哥的活宝妹会折腾狠久,傻傻瓣不清楚,刚才感觉没多久就解决问题了。。。
\item 亲爱的表哥的活宝妹,今天上午的两小时【10-12am】: \textbf{【想要回去看一下停放四个月的ET 框架,深切感受一下,暂停四个月的能量自增长!!!】}
\item 今天早上醒不来,一脚踏进洗手间,就知道,亲爱的表哥的活宝妹住处的楼上破烂贱鸡、贱畜牲的噪音机、各种噪音机,又开了一整夜,吵死人不偿命。今天的状态并不好,可是能够把先前无法连接的狠多细节弄清楚。还算不错。
\end{itemize}
\item 下午,再读一会儿ET 框架里的源码。
\item 如果感觉困意来袭,就去找和实现项目。现在就去找,实现一些项目的思路。下午弄三个小时,4:30pm 左右,去取亲爱的表哥的活宝妹的隐形镜片,和补一张支票。鬼知道,春天系里会肿么样呢?
\end{itemize}
\subsubsection{Mac-M1-M2 游戏项目的【构建环境配置】}
\label{sec-2-1-1}
\begin{itemize}
\item \textbf{【VSC 程序集跳转、构建环境】}: 这个IDE 是目前亲爱的表哥的活宝妹最喜欢用的 IDE. 要把程序集构建好,至少到可以方便跳转各种类、函数的定义与使用等,方便阅读源码。
\item \textbf{【Visual Studio 构建环境】} :
\end{itemize}
\subsubsection{ET 框架:【亲爱的表哥的活宝妹,任何时候,亲爱的表哥的活宝妹就是一定要、一定会嫁给活宝妹的亲爱的表哥!!!爱表哥,爱生活!!!】}
\label{sec-2-1-2}
\begin{itemize}
\item 从现在开始,亲爱的表哥的活宝妹给ET 框架里,还 \textbf{不懂的地方打标记:【TODO】:} 以后亲爱的表哥的活宝妹,只搜索这些标记,就可以一点一点儿把先前不懂的全弄明白了!!
\item 以前,这个框架,亲爱的表哥的活宝妹,感觉可能理解困难的是哪些模块: \textbf{【网络模块】、【一个游戏框架,11-27 个构建项目的、项目间联系、构建逻辑】,【数据库】,【框架服务端,与各项目,各小服,配置逻辑】和【框架里、封装的异步任务的 Coroutine 成员,总感觉没读懂!】【在亲爱的表哥的活宝妹,重构【拖拉机双升游戏】时,重构过程中涉及到的游戏框架四五大主要程序域的划分上,如安卓MVVM 般,理解透彻 et 四第8 章节12 小节】【先前意识到过:服务端各物理机、各小服启动的命令行,或是配置文件行参数】} 等一一弄懂。把这些先前,感觉有点儿困难的快速捡一遍
\end{itemize}
\subsubsection{安卓:内存管理相关,阅读}
\label{sec-2-1-3}
\begin{itemize}
\item 这个先前不曾理解透彻的模块,现在 feels-really-good-to-tackle-it. 现在可以轻松理解透彻。改天需要,知识点总结好;另,内存检测工具等,自己手动实现体会一下。
\item Android性能优化之内存优化:这个据说是什么最高、最深入级别【炼狱级别的?】的整理,但是感觉也就那样。亲爱的表哥的活宝妹消化理解没困难,就是要手动一遍才好。
\item 今天暂时不看了,但是今天晚上回住处休息前,希望把今天看过的安卓这块,文献整理到位,方便亲爱的表哥的活宝妹自己再复习。
\end{itemize}
\subsection{【12/19: Tuesday】:}
\label{sec-2-2}
\begin{itemize}
\item 早上先看安卓内存管理相关的,再回去接着昨天早上的读源码。
\item 亲爱的表哥的活宝妹,玩一会儿,然后 \textbf{【上午一个小时10:27-11:32am,跟昨天一样,再看一小时框架,捡漏,看两小时可以再收获哪些?】} 。亲爱的表哥的活宝妹,新新人类的亲爱的表哥的活宝妹,总是能够从网络上了解、学习到,它们,大家都在用什么?亲爱的表哥的活宝妹,要紧跟大家的研究兴趣与使用流利。。。
\item 昨天晚上最初迷迷糊糊感觉快睡着了;可是因为怕被冻感冒,薄厚三件羽绒服加被子上,热得无法入睡;撤走一两件,终于能够勉强入睡;今天早上就醒不来呀。。闹钟闹了好多遍才起来。。中午大睡了要。。睡了狠久。可以考虑晚上早点儿休息,早上多睡一会儿,还是保障必要必备的休息。 \textbf{【亲爱的表哥的活宝妹,任何时候,亲爱的表哥的活宝妹就是一定要、一定会嫁给活宝妹的亲爱的表哥!!!爱表哥,爱生活!!!】}
\item \textbf{【亲爱的表哥的活宝妹,任何时候,亲爱的表哥的活宝妹就是一定要、一定会嫁给活宝妹的亲爱的表哥!!!爱表哥,爱生活!!!】}
\begin{itemize}
\item 从以前楗不出 pyim 等的各种动态库、现在亲爱的表哥的活宝妹都还算能把它们瓣通来看,现在算是基本可以尝试了。
\end{itemize}
\end{itemize}
\subsection{【12/20: Wednesday】:}
\label{sec-2-3}
\begin{itemize}
\item 昨天的收获不是狠多;昨天晚上,因为亲爱的表哥的活宝妹下午给活宝妹的妈打电话里,能够体会的亲爱的表哥的活宝妹的妈的“情绪”——老人家身边的任何子女都“被禁”哪怕踏入活宝妹的妈的住处半步!亲爱的表哥的活宝妹,昨天晚上没能及时入睡;然后亲爱的表哥的活宝妹,就再次观察体会、亲爱的表哥的活宝妹住处的楼上破烂贱鸡、贱畜牲破烂房东的、极恶手艺匠人、再次挖坑打洞,把室外噪音故意传至亲爱的表哥的活宝妹现摆放床位置的床头!这个世界上还有比亲爱的表哥的活宝妹住处的破烂房东、破烂猪皮肥肉、千斤器嚣鼎、万斤秤砣之流的死肥猪、猪八戒、贱鸡、贱畜牲,更贱恶的存在吗?
\item 亲爱的表哥的活宝妹,没有任何其它想法。只是需要尽力做活宝妹的妈老人家的思想工作。亲爱的表哥的活宝妹,坐守活宝妹的亲爱的表哥的身边 500 年,就是要等一张结婚证;但同时,亲爱的表哥的活宝妹,要给活宝妹的妈的思想工作做好,用积极的心态,来面对所有身处中国大陆的子女,都被胁迫的、一个老人家的“孤独终老”,与用被动的心态,结果不一样。亲爱的表哥的活宝妹,要为妈做好各项思路工作。 \textbf{【亲爱的表哥的活宝妹,任何时候,亲爱的表哥的活宝妹就是一定要、一定会嫁给活宝妹的亲爱的表哥!!!爱表哥,爱生活!!!】}
\item 昨天的收获不是狠多,今天状态不够,但要努力,希望今天能够达到前天的结果。
\item 花时间来解决: \textbf{【Macbook 下的ET 框架的构建、运行环境】} 。只有真正把这些运行环境弄通了,亲爱的表哥的活宝妹才能随心所愿地运行各种示范项目。最迟最天,或今天晚上。
\item 今天上午的两小时:看【拖拉机双升游戏】源码。它是一个 windows-form 表单桌面纸牌小游戏,适合ET 框架重构。亲爱的表哥的活宝妹,觉得,之所以适配、重构一个ET 框架游戏,显得相对困难,可能也是亲爱的表哥的活宝妹,对这个游戏的逻辑,理解得还不是太透彻。游戏的玩法亲爱的表哥的活宝妹懂得,自己玩了狠久。但是游戏的开发者,相对草包,不使用OOD/OOP 设计,所以亲爱的表哥的活宝妹现在读起这个游戏的源码来,就是小蚂蚁掉进团团棉花般,腿爬断也不明白,源码作者写的是什么糊糊东西。。。想上午花两小时,把这个项目看透。才方便接下来重构适配ET 框架。 \textbf{【亲爱的表哥的活宝妹,任何时候,亲爱的表哥的活宝妹就是一定要、一定会嫁给活宝妹的亲爱的表哥!!!爱表哥,爱生活!!!】}
\end{itemize}
\subsection{【12/21: Thursday】:}
\label{sec-2-4}
\begin{itemize}
\item 【颅内血管性头痛,感染风寒,养病,玩儿】效率低,只简单看了看。 \textbf{亲爱的表哥的活宝妹,从昨天自己网络查清【颅内血管有血栓、血管狭窄性】受寒头痛,开始,今天一段时间的养生重点就是清脑血管里的堵塞与血栓。想要遍试网络上的偏方,找到可以解决亲爱的表哥的活宝妹头痛问题的办法。}
\item 拖拉机双升的源码,还没能完全读懂
\item \textbf{【安卓多进程应用】,多进程模块多JVM 虚拟机涉及的问题与注意事项,纯理论} 。还应该需要 GitHub 上抓八个十个项目下来,读它们的源码,多看几个案例,或才能真正明白。现瓣手指头数,几个注意事项,主要包括,与多进程各进程间相互独立的内存相关的,相对于单进程应用失效的方面:
\begin{itemize}
\item \textbf{Static 静态变量、Singleton 单例模式失效}
\item \textbf{SharedPreference 失效} 。虽然它是写配置到一个文件,但原理上是,写入文件前进程内存里存在 dirty 缓存,所以会导致多进程间无法同步
\item 同享文件同步读写失效?
\item \textbf{多进程应用的、多次一一启动} 。解法是 OnCreate() 逻辑同样模块化。只主进程启动应用,其它模块相生相伴或安静退出。。
\item 多进程的创建, \textbf{安卓下的 zygote fork() 有哪些适用场景?}
\item 一个 \textbf{应用多模块,多进程了,【冷温热启动】以及太多视图的、可能会必要的【延迟加载】等,原理,也需要再学习总结一下} 。
\end{itemize}
\item 安卓多进程通信的几种方式、优缺点,与适用场景等。
\end{itemize}

\includegraphics[width=.9\linewidth]{./pic/readme_20231221_162825.png}
\subsection{【12/2: 】:}
\label{sec-2-5}
\subsection{【12/2: 】:}
\label{sec-2-6}
\subsection{【12/2: 】:}
\label{sec-2-7}

\section{A Joke of the Fall Semester: Big Advertisement for Advisor\ldots{}\ldots{}!!!}
\label{sec-3}
\begin{itemize}
\item 亲爱的表哥的活宝妹的破烂贱鸡、贱畜牲般一再发疯犯贱的、破烂导师的一学期一次的指导:鸡蛋里挑骨头,高射炮打蚊子,小题大作,呵呵呵呵呵。。。。。
\item \textbf{【亲爱的表哥的活宝妹,任何时候,亲爱的表哥的活宝妹就是一定要、一定会嫁给活宝妹的亲爱的表哥!!!爱表哥,爱生活!!!】}
\item In summary, an advisor defensive summarized to defend himself after 1st PhD Semester, by claiming he was travelling without offering any necessary suggestions -- A dirty blood war battling\ldots{}..
\end{itemize}
\subsection{【Background】:ever introduced or misleading on propose ?}
\label{sec-3-1}

\subsection{【Lack of research progress】 vs 【Lack of any guidence, nor monitoring from department】}
\label{sec-3-2}

\subsection{【Noncompliance】: from all parties -- Student, Advisor, Department\ldots{}!!}
\label{sec-3-3}
\begin{itemize}
\item Both parties: Advisor and I reported to department.
\item Department organized 1 meeting, but \textbf{Advisor was the one NOT complianced to Department's suggestions offered during the meeting firstly.}
\item By reporting to Department, \textbf{I suggest and request TRANSPARENT MANAGEMENT from Department and monitoring on the Advisor and student mentoring relationship}, but \textbf{advisor and department play the game of escaping by applying means of underwater mis-manipulations, bluring boundaries, for example, played treats multiple times by department secretary.}
\item As a student, \textbf{I do NOT have these requirements clarified, but disputed by a mean advisor.}
\end{itemize}

\subsection{【Poor interpersonal skills \& TA performance】vs 【Poor advisor's personality \& Poor Department Administration】}
\label{sec-3-4}
\subsubsection{Advisor:}
\label{sec-3-4-1}
\begin{itemize}
\item executes himself by lying on propose to un-guide a student;
\item Even worse, misleading on propose by requiring another Phd student reproting weekly progress by stating on reading, never any implementation.
\item Misleading on propose has been manipulated by this Advisor during multiple secenario multiple times.
\item Personality treaky, unsincere.
\end{itemize}
\subsubsection{Department:}
\label{sec-3-4-2}
\begin{itemize}
\item 
\end{itemize}
\section{【亲爱的表哥的活宝妹,任何时候,亲爱的表哥的活宝妹,就是一定要、一定会嫁给活宝妹的亲爱的表哥!!!爱表哥,爱生活!!!】}
\label{sec-4}
\section{}
\label{sec-5}
% Emacs 29.1 (Org mode 8.2.7c)
\end{document}