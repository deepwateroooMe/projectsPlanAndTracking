% Created 2023-12-17 Sun 22:17
\documentclass[9pt, b5paper]{article}
\usepackage{xeCJK}
\usepackage[T1]{fontenc}
\usepackage{bera}
\usepackage[scaled]{beraserif}
\usepackage[scaled]{berasans}
\usepackage[scaled]{beramono}
\usepackage[cache=false]{minted}
\usepackage{xltxtra}
\usepackage{graphicx}
\usepackage{xcolor}
\usepackage{multirow}
\usepackage{multicol}
\usepackage{float}
\usepackage{textcomp}
\usepackage{algorithm}
\usepackage{algorithmic}
\usepackage{latexsym}
\usepackage{natbib}
\usepackage{geometry}
\geometry{left=1.2cm,right=1.2cm,top=1.5cm,bottom=1.2cm}
\usepackage[xetex,colorlinks=true,CJKbookmarks=true,linkcolor=blue,urlcolor=blue,menucolor=blue]{hyperref}
\newminted{common-lisp}{fontsize=\footnotesize} 
\author{deepwaterooo}
\date{\today}
\title{Winter 3 Weeks Holidays: Basic Plan + Tracking}
\hypersetup{
  pdfkeywords={},
  pdfsubject={},
  pdfcreator={Emacs 29.1 (Org mode 8.2.7c)}}
\begin{document}

\maketitle
\tableofcontents


\section{Basic Plan: \textbf{【亲爱的表哥的活宝妹,任何时候,亲爱的表哥的活宝妹就是一定要、一定会嫁给活宝妹的亲爱的表哥!!!爱表哥,爱生活!!!】}}
\label{sec-1}
\begin{itemize}
\item \textbf{【夏天两三个项目:】} 项目暂停、时间再过四个月,亲爱的表哥的活宝妹,明天再看,能否无师自通、自己开各种窍窍呢?!!明天再看暂停了四个月的夏天两三个项目,亲爱的表哥的活宝妹,会否如先前算法般、再次体会、项目上的成长进步?
\begin{itemize}
\item 时间转得狠快,转眼四个月眨眼就过去了,亲爱的表哥的活宝妹,夏天的两三个项目,就因为亲爱的表哥的活宝妹先前完全不懂的 linux 操作系统,被放去休息四个月。
\item 不是亲爱的表哥的活宝妹,挫败,只是先前完全不会操作系统的亲爱的表哥的活宝妹,没能挤出时间精力去弄。
\item 亲爱的表哥的活宝妹,自两年前的 8 月,开始真正从头、从第一个题目写算法,那个时候,实在是傻傻不知道怎么想题目;可是半年、大半年后,亲爱的表哥的活宝妹,只要休息好,只要能够自己把思路想明白,亲爱的表哥的活宝妹,就是有绝对的自信,一定可以修改掉过程中边边角角的小错误与偏差,把题目完全解出来、通过所有测试。 \textbf{亲爱的表哥的活宝妹,自己,从算法的练习里,建立过编程与解题的极彻底、让亲爱的表哥的活宝妹自己动容的自信。亲爱的表哥的活宝妹,觉得,时间是太神奇的催化剂,亲爱的表哥的活宝妹也女大十八变,亲爱的表哥的活宝妹的算法,原来也可以、也会无师自通!!}
\item 时间转得狠快,转眼四个月眨眼就过去了,亲爱的表哥的活宝妹,夏天的两三个项目,暂停四个月,亲爱的表哥的活宝妹,现在要继续弄它们前的第一想法,这三个周一定会再回去看,写得完善些。 \textbf{动手前、下口前,这一次,亲爱的表哥的活宝妹,先想,是否会体会如先前、去年工作后体会过的,亲爱的表哥的活宝妹,又女大十八变般,又无师自通了呢?接下来几天检测一下!!}
\end{itemize}
\item \textbf{【Mac-M1-M2 项目环境配置:】} 主要是 \textbf{C\# 游戏框架相关的项目、配置需求}
\begin{itemize}
\item 亲爱的表哥的活宝妹的M2, 上半年还没能真正适应,各种项目的运行,基本仍然是在破烂快死掉的 Windows 10 系统上。
\item 主要原因也是: \textbf{中文社区用苹果的人少,中文社区的必要科普、普及不够。英文社区,在亲爱的表哥的活宝妹这里,可能,还存在一定程度上的语言障碍、与消化理解困难。}
\item 那个时候,亲爱的表哥的活宝妹, \textbf{不懂得、不知道、该怎么配置必要的项目环境。}
\item 现在,买电脑快1 年了(10 个月),亲爱的表哥的活宝妹,觉得, \textbf{现在,接下来三个周,应该是时候,广泛搜索,为亲爱的表哥的活宝妹自己科普、学习与深入,Mac-M1-M2 里,亲爱的表哥的活宝妹所需要的、必要的项目运行环境配置。}
\item 若能够 \textbf{把 Rime M1M2 环境下的输入法,自己编译安装进电脑,也算是,一个大进步} 。先前,亲爱的表哥的活宝妹,连 emacs-pyim-librime 动态库都建不出来。。。
\end{itemize}
\item \textbf{【实现几个小项目:】} 推 deepwateroooMe-github 上面去
\begin{itemize}
\item 去年离职前,亲爱的表哥的活宝妹的 deepwaterooo-github 帐户,被他们锁死弄没了。亲爱的表哥的活宝妹,不得不重建一个 deepwateroooMe-github 帐户,可是仓库少了太多。 \textbf{亲爱的表哥的活宝妹先前的 deepwaterooo-github 帐户,纪录着亲爱的表哥的活宝妹专业成长足迹的、几乎所有仓库,都不见了。极其可惜。}
\item 亲爱的表哥的活宝妹,上个周,因为代课老师要求,打算 fork-xv6-public 时,发现无法顺利登录 deepwateroooMe-github 帐户。担心过,不过还好,今天看它还是活的!
\item 亲爱的表哥的活宝妹,觉得 \textbf{上面的项目太少了, 50 个仓库,还有狠多是 fork 他人的} 。觉得 \textbf{只看相对复杂一点儿的大型游戏框架,就显得自己设计与实现的项目太少,而真正的基本功训练也不可少。想要再添加、调序与实现几个项目,推上去。} 现在只是不清楚,是实现(哪怕是修改他人的原始 naive 版本项目)C\# 游戏项目,还是设计实现几个安卓项目?想想,希望这三个周添加几个。
\begin{itemize}
\item \textbf{C\# 游戏项目} :可能找 github 上的幼稚项目,改成亲爱的表哥的活宝妹,希望的样子!!
\item \textbf{安卓项目} :就需要亲爱的表哥的活宝妹,自己设计、实现,难度上应该到中高层次,才可以。
\end{itemize}
\end{itemize}
\item \textbf{【google-play 上,希望尽快上1个小游戏,走一遍流程:】}
\item \textbf{【Unity 里的学习,确定一个方向】} :Unity 游戏引擎,涉及到的知识面比较广了。上半年主要弄了些游戏框架里网络相关的部分,还不一定消化理解得透彻。亲爱的表哥的活宝妹,应该 \textbf{需要,为自己在个这个引擎上的学习,确定几个自己能够理解消化得透彻、真正感兴趣、甚至将来可以作为自己工作内容的方向。}
\begin{itemize}
\item \textbf{【ARM-Linux 系统,与内存管理】} :这个学期,亲爱的表哥的活宝妹的【操作系统】相关的基础知道补习,还算是狠彻底的,主要是借助亲爱的表哥的活宝妹的舅舅的课上的作业与考试。
\item \textbf{【安卓内存优化】,与【Unity 内存管理,与优化】} 相关的,可以再复习一遍,复习,总结,也寻找自己感兴趣的版块与方向。
\end{itemize}
\item \textbf{【被广泛接纳、使用的、第三方库,的学习:】} 需要涉及、遍布一定程度上的广度,扩充知识面。
\begin{itemize}
\item \textbf{【emacs-org-mode 工具的启发】} :
\begin{itemize}
\item 上半年的 pyim, 是多少年来,亲爱的表哥的活宝妹,第一次真正实现 emacs 里、自己的中文输入。多少年前,IDE 还不先进不完善,亲爱的表哥的活宝妹自己安装 linux 系统,自己配置 linux系统里 latex 相关的环境,字体等,那个时候,它是每台新电脑、笔记本最头痛的难题,不死去活来折腾好几天,亲爱的表哥的活宝妹,都配置不出来。现在都“傻瓜相机”化了,不再头痛。
\item \textbf{亲爱的表哥的活宝妹,是如何,把自己变成 emacs 文本编辑器 org-mode 相关配置的,【中高端】水平的?}
\item 不是想骄傲,为自己迷惑、寻找方向时,提供一个视角。 \textbf{在安卓,C\# 游戏相关版块,亲爱的表哥的活宝妹,要如何借鉴,这些几往的、过往的经验与教训,能够在可以找到的有兴趣的方向上,学习得更深入一点儿?} 最近可以多想想这个问题。
\end{itemize}
\item \textbf{【安卓】} :
\begin{itemize}
\item \textbf{【图库 glide?】} :去年秋天?,他们提过一次,安卓图片库 glide? 里,关于动漫相关的功能模块的完善,可是,那个时候,亲爱的表哥的活宝妹,草草了结了小测试项目,没能理解透彻,或是受到启发,没兴趣。回来可以再看一次。
\item \textbf{【相机、视频相关的】} ,中文社区,感觉流行过了,不再热了。可是基本功,可以自己学习一下
\end{itemize}
\item C\# 游戏相关:感觉,就是那些,各种不同的, \textbf{游戏框架} 了?
\end{itemize}
\item \textbf{【游戏:苹果端,打包、构建流程,熟悉掌握】} :
\begin{itemize}
\item 亲爱的表哥的活宝妹,先前,只做安卓端。现在,基本环境都具备,是否应该也熟悉一下游戏苹果端的构建、打包相关的流程。毕竟,弄个游戏出来只走安卓端。。
\end{itemize}
\item \textbf{【Emacs org-mode IEEE article automate configurations:】} 中文社区,科研技术落后, emacs 这类命令式编辑器也不流行,用 emacs-org-mode 来配置 IEEE-article 自动化的,就几乎找不到相关的配置。英文社区有、存在,但是真正牛的分享贴,不包括所有细节;亲爱的表哥的活宝妹,有兴趣想配置,可是某些边边角角找不到参考,倾向于,可能会【行百里者,半于九十】。得真正花些时间、精力,才能把这个真正配置出来,到可以完全使用 emacs org-mode 来 100\% 完成所有 summary, 而不需要使用 latex 的程序。
\item \textbf{【emacs-pyim-librime】} :亲爱的表哥的活宝妹,现在已经可以构建M1M2 下在的动态包裹。还 \textbf{需要完善一下,中文输入的词库自动化同步} 。哪天花点儿时间,可以简单解决。
\item 【:】 \textbf{其它的、想到再添加。【亲爱的表哥的活宝妹,任何时候,亲爱的表哥的活宝妹就是一定要、一定会嫁给活宝妹的亲爱的表哥!!!爱表哥,爱生活!!!】}
\item \textbf{【emacs skim legend bug】}: 
\begin{itemize}
\item 亲爱的表哥的活宝妹的放了 1.5 年的 \textbf{【emac skim export BUG:】} ,不曾真正花时间去解决: \textbf{为什么 emacs export pdf SKIM 里永远需要,亲爱的表哥的活宝妹连点三次 enter?得自动化把这三次点击去掉。}
\item 现在,Skim 更新了,终于不用,亲爱的表哥的活宝妹,连点三次 enter 键了?!!
\end{itemize}
\end{itemize}
\section{Weeks Updates + Recordings/Trackings}
\label{sec-2}
\begin{itemize}
\item 【:】for a productive winder holiday!! Happy Holidays!
\end{itemize}
% Emacs 29.1 (Org mode 8.2.7c)
\end{document}